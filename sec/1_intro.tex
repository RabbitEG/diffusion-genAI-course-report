\section{Introduction}

Text-to-image diffusion models such as Stable Diffusion \cite{rombach2022high} have demonstrated remarkable capabilities in generating high-quality images from textual descriptions. However, precise spatial control over the generated content—such as specifying exact poses, edges, or depth layouts—remains challenging when relying solely on text prompts. ControlNet \cite{zhang2023adding} addresses this limitation by introducing a trainable copy of the encoder blocks that processes structural conditions (e.g., Canny edges, human poses, depth maps) and modulates the frozen backbone through zero-initialized convolutions.

While the original ControlNet paper establishes the general framework, several practical questions arise when applying this approach to domain-specific applications with limited training data:

\begin{itemize}
    \item Should the Stable Diffusion backbone remain frozen during training, or can selective unfreezing improve adaptation to new condition distributions?
    \item Is it beneficial to inject conditional signals at all decoder levels, or does restricting injection to specific layers yield better generalization?
    \item How do these choices interact with each other and with dataset scale?
\end{itemize}

In this paper, we present a systematic ablation study addressing these questions. Using the Fill50K dataset as a testbed, we evaluate four configurations spanning the combinatorial space of \texttt{sd\_locked} $\in \{\text{True}, \text{False}\}$ and \texttt{only\_mid\_control} $\in \{\text{True}, \text{False}\}$. Our key contributions are:

\begin{enumerate}
    \item \textbf{Empirical characterization} of convergence patterns, generation quality, and overfitting tendencies across configurations.
    \item \textbf{Identification of Configuration D} (\texttt{sd\_locked=False}, \texttt{only\_mid\_control=True}) as optimal for small-data scenarios, achieving high-quality generation at approximately 2400 steps.
    \item \textbf{Mechanistic analysis} explaining the observed behaviors through the lens of effective capacity, gradient pathway constraints, and loss landscape geometry.
    \item \textbf{Practical guidelines} for training ControlNet variants under data-limited conditions.
\end{enumerate}

%==============================================================================
% 2. PROBLEM FORMULATION
%==============================================================================
