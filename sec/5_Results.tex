\section{Experimental Results}
\label{sec:results}

\subsection{Configuration Comparison}

Table~\ref{tab:configs} summarizes the four experimental configurations and their key characteristics.

\begin{table}[h]
\centering
\small
\caption{Ablation configurations and observed outcomes.}
\label{tab:configs}
\begin{tabular}{@{}ccccc@{}}
\toprule
Config & \texttt{sd\_locked} & \texttt{only\_mid} & Steps & Quality \\
\midrule
A & True & True & 2700 & Medium \\
B & False & False & -- & Low (Overfit) \\
C & True & False & 1500 & Overfit \\
D & False & True & 2400 & \textbf{High} \\
\bottomrule
\end{tabular}
\end{table}

\subsection{Detailed Analysis by Configuration}

\textbf{Configuration A}: This conservative setting achieves medium quality with stable convergence around 2700 steps. The frozen backbone ensures prior preservation, while middle-only injection provides implicit regularization. However, the inability to adapt backbone features limits conditional alignment quality.

\textbf{Configuration B}: Full unfreezing with multi-level injection dramatically increases effective capacity. While training loss decreases rapidly, validation metrics degrade, indicating severe overfitting. The model memorizes training condition-output pairs rather than learning generalizable mappings in practice.

\textbf{Configuration C}: Despite keeping the backbone frozen, multi-level injection still leads to overfitting, with convergence as early as 1500 steps followed by quality degradation. This suggests that injection depth, not just backbone training, contributes to overfitting risk.

\textbf{Configuration D}: This configuration achieves the best results. The unfrozen backbone allows adaptation to the condition distribution, while middle-only injection constrains the effective update subspace. Emergent convergence occurs around 2400 steps, producing high-quality outputs with strong generalization.

\subsection{Emergent Convergence Phenomenon}

A notable observation across configurations is the non-linear convergence behavior. Models maintain outputs close to the unconditional baseline for extended periods before exhibiting sudden alignment with input conditions. This ``emergent convergence'' is most pronounced in Configuration D: \textbf{Steps 0--1500} show minimal deviation from pretrained behavior, \textbf{Steps 1500--2200} show a gradual increase in condition sensitivity, \textbf{Steps 2200--2400} show rapid alignment with structural conditions, and \textbf{Steps 2400+} yield stable high-quality generation.

Configuration B also shows rapid change but transitions into overfitting rather than stable alignment.

%==============================================================================
% 6. DISCUSSION
%==============================================================================
